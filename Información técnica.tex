\documentclass{article} % Tamaño de fuente 12pt

% \usepackage[utf8]{inputenc} % Codificación de caracteres
\usepackage[spanish]{babel} % Idioma español
\usepackage{graphicx} % Para incluir imágenes
\usepackage{hyperref} % Para enlaces (opcional)
\usepackage{comment} % Para comentarios
\usepackage{enumitem} % Para negrita en listas
\usepackage[letterpaper, left=1in, right=1in, top=1.4in, bottom=1in, headheight=2in]{geometry}
\usepackage{fancyhdr} % Para incluir imágenes en el encabezado

% Información básica
\title{Documentación del Proyecto de Horarios}
\author{Janer Alberto Vega Jácome}
\date{\today}

\fancyhf{} % Limpia los encabezados y pies de página
\renewcommand{\headrulewidth}{0pt} % Elimina la línea horizontal del encabezado
\fancyhead[L]{
%    \vspace{-.5cm}
    \begin{minipage}{0.3\textwidth}
        \centering
        \includegraphics[height=1.5cm]{Imágenes/logoEscuela.png}
    \end{minipage}
}
\fancyhead[C]{
%    \vspace{-.5cm}
    \begin{minipage}{0.4\textwidth}
        \centering
        \textbf{\Large Escuela de Ingeniería de Sistemas e Informática} \\
        \textit{Universidad Industrial de Santander}
    \end{minipage}
}
\fancyhead[R]{
%    \vspace{-.5cm}
    \begin{minipage}{0.3\textwidth}
        \centering
        \includegraphics[height=1.5cm]{Imágenes/logoUIS.png}
    \end{minipage}
}

\fancyfoot[C]{\thepage} % Número de página centrado

\pagestyle{fancy}

\begin{document}

\maketitle % Crea la portada con título, autor y fecha

    \section{Arquitectura del sistema}
    \subsection{Decripción}
    \noindent El sistema de gestión de horarios sigue una arquitectura cliente-servidor, donde el frontend y el backend están separados y se comunican a través de una API REST. Adicionalmente, se utiliza una base de datos para el almacenamiento persistente de los datos.
    
    \begin{enumerate}[font=\bfseries]
            \item \textbf{Angular:} El frontend se desarrolla utilizando el framework Angular, que organiza el código en componentes, servicios y módulos.            
            \begin{itemize}
                \item \textbf{Componentes:} Representan las diferentes partes de la interfaz de usuario (vistas de horarios, vista de login, etc.).
                \item \textbf{Servicios:} Módulos que manejan la lógica de negocio en el frontend y se comunican con el backend a través de HTTP.
                \item \textbf{Módulos:} Agrupan componentes y servicios relacionados.
            \end{itemize}
        
            \item \textbf{Spring Boot:} El backend se construye con Spring Boot, que sigue una arquitectura de capas:
            \begin{itemize}
                \item \textbf{Controladores:} Reciben las solicitudes del frontend, interactúan con los servicios y devuelven las respuestas.
                \item \textbf{Servicios:} Implementan la lógica, como la generación de horarios, la validación de datos y el acceso a la base de datos.
                \item \textbf{Repositorios:} Se encargan de la interacción con la base de datos, utilizando JPA (Java Persistence API) para mapear las entidades del modelo de datos a tablas en la base de datos.
                \item \textbf{Entidades:} Clases que representan las tablas en la base de datos y son utilizadas para mapear los datos entre la aplicación y la base de datos.
            \end{itemize}
        
        \item \textbf{Base de datos} \\
        \noindent Se utiliza una base de datos MySQL para almacenar los datos del sistema, como información de usuarios, asignaturas, profesores, disponibilidad horaria, aulas y horarios.
        
        \item \textbf{Integración y comunicación entre las capas}
        \begin{itemize}
            \item \textbf{API RESTful:} El backend de Spring Boot expone una serie de endpoints RESTful que el frontend en Angular consume. Esta comunicación se realiza a través de HTTP, donde el frontend envía solicitudes y recibe respuestas en formato JSON.
            \item \textbf{Seguridad:} La autenticación y autorización se manejan en la capa de negocio, asegurando que solo usuarios autenticados puedan acceder a ciertos recursos, utilizando \texttt{JWT} (JSON Web Tokens).
        \end{itemize}
    \end{enumerate}
    
    \subsection{Módulos principales}
    \begin{itemize}
        \item \textbf{Frontend:} 
        \begin{itemize}
            \item \textbf{Módulo de interfaz de usuario (UI)}
                \begin{itemize}
                    \item \textbf{Descripción:} Este módulo maneja la interfaz de usuario de la aplicación. Su propósito es permitir que los usuarios interactúen con el sistema de manera amigable y eficiente.
                    \item \textbf{Responsabilidad:} Mostrar datos y capturar las entradas del usuario.
                    \item \textbf{Restricciones:} No tiene acceso directo a la lógica de negocio o a la base de datos; solo interactúa con los datos que le son provistos a través del módulo de comunicación.
                    \item \textbf{Dependencias:} Depende del \textbf{\emph{Módulo de comunicación HTTP}} para obtener y enviar datos al backend.
                    \item \textbf{Implementación:} Este módulo se implementa en archivos \texttt{.html}, \texttt{.ts} y \texttt{.css} dentro de la carpeta \texttt{src/app}, dividido en varios componentes.
                \end{itemize}
            \item \textbf{Módulo de comunicación HTTP}
                \begin{itemize}
                    \item \textbf{Descripción:} Este módulo maneja todas las solicitudes HTTP que el frontend envía al backend. Su objetivo es interactuar con el backend para obtener y enviar datos.
                    \item \textbf{Responsabilidad:} Enviar solicitudes HTTP y procesar las respuestas.
                    \item \textbf{Restricciones:} No debe contener lógica de presentación ni lógica de negocio, solo debe encargarse de la comunicación.
                    \item \textbf{Dependencias:} Depende del \textbf{\emph{Módulo UI}} para recibir solicitudes de datos y enviar respuestas, y de los servicios REST expuestos por el backend en Spring Boot.
                    \item \textbf{Implementación:} Este módulo se implementa en archivos \texttt{.ts} dentro de la carpeta \\\texttt{src/app/services}.
                \end{itemize}
        \end{itemize}
        
        \item \textbf{Backend:} 
        \begin{itemize}
            \item \textbf{Módulo de controladores}
            \begin{itemize}
                \item \textbf{Descripción:} Este módulo actúa como intermediario entre el frontend y la lógica de negocio. Maneja las solicitudes HTTP entrantes y devuelve las respuestas adecuadas.
                \item \textbf{Responsabilidad:} Recibir solicitudes HTTP, delegarlas a los servicios adecuados, y retornar respuestas.
                \item \textbf{Restricciones:} No debe contener lógica de negocio; solo delega tareas a los servicios.
                \item \textbf{Dependencias:} Depende del \textbf{\emph{Módulo de servicios}} para realizar la lógica de negocio y del \textbf{\textit{Módulo de comunicación HTTP}} para recibir solicitudes desdel el frontend.
                \item \textbf{Implementación:} Este módulo se implementa en clases Java con anotaciones \\\texttt{@RestController}, ubicadas en el paquete \texttt{uis.horariouis.controller}.
            \end{itemize}
            
            \item \textbf{Módulo de servicios}
            \begin{itemize}
                \item \textbf{Descripción:} Este módulo contiene la lógica de negocio de la aplicación. Es responsable de procesar los datos y aplicar las reglas de negocio antes de enviarlos a los controladores o repositorios.
                \item \textbf{Responsabilidad:} Implementar la lógica de negocio y realizar transformaciones en los datos.
                \item \textbf{Restricciones:} No debe interactuar directamente con la interfaz de usuario ni con la base de datos; estas interacciones deben pasar por los controladores y repositorios, respectivamente.
                \item \textbf{Dependencias:} Depende del \textbf{\emph{Módulo de repositorios}} para acceder a la base de datos.
                \item \textbf{Implementación:} Este módulo se implementa en clases Java dentro del paquete \\\texttt{uis.horariouis.service}.
            \end{itemize}
            
            \item \textbf{Módulo de repositorios}
            \begin{itemize}
                \item \textbf{Descripción:} Este módulo maneja la persistencia de los datos en la base de datos. Proporciona abstracciones para realizar operaciones CRUD sobre la base de datos.
                \item \textbf{Responsabilidad:} Interactuar con la base de datos a través de JPA para realizar operaciones CRUD.
                \item \textbf{Restricciones:} No debe contener lógica de negocio, solo acceso a datos.
                \item \textbf{Dependencias:} Depende del \textbf{\emph{Módulo de gestión de datos}} en MySQL para almacenar y recuperar datos.
                \item \textbf{Implementación:} Este módulo se implementa en interfaces de Java que extienden \\\texttt{JpaRepository}, ubicadas en el paquete \texttt{uis.horariouis.repository}.
            \end{itemize}
            
            \item \textbf{Módulo de seguridad}
            \begin{itemize}
                \item \textbf{Descripción:} Este módulo se encarga de la autenticación y autorización de usuarios dentro de la aplicación.
                \item \textbf{Responsabilidad:} Proteger los recursos de la aplicación mediante autenticación y autorización.
                \item \textbf{Restricciones:} No debe gestionar la lógica de negocio, solo asegurar que las solicitudes sean realizadas por usuarios autenticados y autorizados.
                \item \textbf{Dependencias:} Depende de \texttt{\texttt{JWT}} para la autenticación.
                \item \textbf{Implementación:} Este módulo se implementa en clases Java con configuraciones de seguridad, ubicadas en el paquete \texttt{uis.horariouis.security}.
            \end{itemize}
        \end{itemize}
        
        \item \textbf{Base de datos:}
        \begin{itemize}
            \item \textbf{Módulo de gestión de datos}
            \begin{itemize}
                \item \textbf{Descripción:} Este módulo es responsable de gestionar el almacenamiento y recuperación de datos en la base de datos MySQL.
                \item \textbf{Responsabilidad:} Almacenar, recuperar y manipular datos según las operaciones realizadas por el backend.
                \item \textbf{Restricciones:} Solo maneja datos estructurados según el esquema de la base de datos.
                \item \textbf{Dependencias:} Depende de las configuraciones de la base de datos y de la conexión establecida por el \textbf{\emph{Módulo de repositorios}}.
                \item \textbf{Implementación:} Este módulo se implementa en la estructura de la base de datos, incluyendo tablas, índices, y procedimientos almacenados.
            \end{itemize}
        \end{itemize}
    \end{itemize}
    
    \subsection{Implementación}
    \subsubsection{Seguridad en el Frontend}
    \noindent La seguridad en el proyecto se implementa a través de varios mecanismos clave que trabajan juntos para proteger la aplicación y asegurar que solo los usuarios autenticados y autorizados puedan acceder a los recursos sensibles. A continuación, se detalla cómo se han implementado estos mecanismos:
    \begin{enumerate}
        \item \textbf{Autenticación con \texttt{JWT} (JSON Web Tokens)}
        \begin{enumerate}
            \item \textbf{Generación del token:} Cuando un usuario inicia sesión, el backend autentica sus credenciales y genera un token \texttt{JWT}. Este token contiene la información del usuario y su tiempo de expiración, y es firmado digitalmente para evitar manipulaciones.
            \item \textbf{Almacenamiento:} El token \texttt{JWT} generado es enviado al frontend y almacenado en el \\\texttt{localStorage} del navegador bajo la clave \texttt{token\_user}. Este token se utiliza para autenticar todas las solicitudes subsiguientes del usuario.
        \end{enumerate}
        \item \textbf{Interceptores para la seguridad de las solicitudes} \\
        \noindent El interceptor \texttt{authApiInterceptor} agrega automáticamente el token \texttt{JWT} a cada solicitud HTTP que se envía al backend:
        \begin{enumerate}
            \item \textbf{Intercepción de Solicitudes:} El interceptor intercepta cada solicitud \texttt{HTTP} y, si el token \texttt{JWT} está presente en \texttt{localStorage}, lo agrega en el encabezado \texttt{Authorization} de la solicitud.
            \item \textbf{Validación por el Backend:} El backend valida el token en cada solicitud para asegurar que sea válido y que el usuario esté autorizado para acceder al recurso solicitado. Si el token es inválido o ha expirado, el backend rechaza la solicitud con un error de autenticación.
        \end{enumerate}
        \item \textbf{Protección de rutas con Guards} \\
        \noindent Para proteger las rutas dentro de la aplicación y asegurar que solo los usuarios autenticados puedan acceder a ciertas secciones, se utiliza un guard:
        \begin{enumerate}
            \item \textbf{Guard de Autenticación (\texttt{loginGuard}):} Este guard verifica si el usuario tiene un token \texttt{JWT} válido almacenado en el \texttt{localStorage}. Si el token existe, se permite el acceso a la ruta. Si no, el usuario es redirigido a la página de login. \\
        Este guard está implementado en la configuración de rutas (en el archivo \texttt{app.routes.ts}) de la aplicación, asegurando que todas las rutas críticas estén protegidas.
        \end{enumerate}
        \item \textbf{Protección de componentes y datos sensibles}
        \begin{enumerate}
            \item \textbf{Componentes protegidos:} El uso de los componentes críticos como \texttt{ModifyComponent} y \\\texttt{NewComponent} están protegidos mediante el uso de guards, asegurando que solo los usuarios autorizados puedan realizar modificaciones o crear nuevos registros.
            \item \textbf{Manejo Seguro de Datos:} Los datos sensibles, como contraseñas y tokens, son manejados con precaución en toda la aplicación. Las contraseñas nunca se almacenan en texto plano, y los tokens se mantienen en \texttt{localStorage} solo durante la duración de la sesión del usuario.
        \end{enumerate}
        \item \textbf{Respuesta y manejo de errores}\\
        En caso de que una solicitud no sea autenticada correctamente:
        \begin{enumerate}
            \item \textbf{Manejo de errores:} El interceptor también maneja los errores que pueden ocurrir durante la comunicación con el backend. Si ocurre un error de autenticación, se muestra un mensaje de error.
        \end{enumerate}
    \end{enumerate}
    
    \subsubsection{Seguridad en el Backend}
    \noindent El sistema de seguridad está diseñado con Spring Security. La seguridad se implementa mediante el uso de \texttt{JWT} (JSON Web Tokens) para la autenticación y configuración de acceso basada en roles.
    
    \begin{enumerate}
        \item \textbf{Configuración de seguridad}
        \noindent En la clase \texttt{SecurityConfigurer.java} se definen las reglas para la autenticación y autorización:
        \begin{itemize}
            \item \textbf{Autenticación y autorización:}
            \begin{itemize}
                \item Se permite el acceso público a ciertas rutas, como la de autenticación \\\texttt{(/api/security/authenticate)}.
                \item Se restringe el acceso a otras rutas según los roles de usuario \texttt{(USER, ADMIN)}. Solo los usuarios con rol \texttt{ADMIN} pueden realizar operaciones \texttt{POST, PUT, y DELETE}.
                \item Se habilita la autenticación básica \texttt{HTTP}, lo cual es útil para acceder a la documentación Swagger.
            \end{itemize}
            \item \textbf{Gestión de sesiones:}
            \begin{itemize}
                \item El sistema utiliza una política de sesión sin estado \texttt{(STATELESS)}, lo que significa que no se mantiene una sesión en el servidor.
            \end{itemize}
            \item \textbf{Cross-Origin Resource Sharing \texttt{(CORS)}}
            \begin{itemize}
                \item Se configuran las reglas de \texttt{CORS} para permitir que el frontend interactúe con el backend desde orígenes específicos.
            \end{itemize}
        \end{itemize}
        
        \item \textbf{JSON Web Tokens \texttt{(JWT)}}
        \noindent Se utiliza JWT para manejar la autenticación de los usuarios:
        \begin{itemize}
            \item \textbf{Generación y Validación de Tokens:}
            \begin{itemize}
                \item La clase \texttt{JwtUtil.java} se encarga de generar los tokens \texttt{JWT} cuando un usuario se autentica exitosamente. También valida los tokens y extrae información relevante como el nombre de usuario, la fecha de expiración. Puede extraer un reclamo en específico o todos los reclamos del token \texttt{JWT}.
            \end{itemize}
            \item \textbf{Filtro de solicitudes:}
            \begin{itemize}
                \item El filtro \texttt{JwtRequestFilter}.java intercepta cada solicitud \texttt{HTTP} para verificar si contiene un token \texttt{JWT} válido en el encabezado de autorización. Si el token es válido, se establece la autenticación en el contexto de seguridad.
            \end{itemize}
        \end{itemize}
        
        \item \textbf{Detalles del usuario y servicios}
        \begin{itemize}
            \item \textbf{CustomUserDetails.java:}
            \begin{itemize}
                \item Esta clase implementa la interfaz \texttt{UserDetails} de Spring Security y proporciona los detalles de autenticación y autorización del usuario, como el nombre de usuario y la contraseña.
            \end{itemize}
            \item \textbf{CustomUserDetailsService.java:}
            \begin{itemize}
                \item Este servicio se encarga de cargar los detalles del usuario desde la base de datos utilizando el repositorio \texttt{UsuarioRepository}. Si el usuario es encontrado, se retornan sus detalles encapsulados en un objeto \texttt{CustomUserDetails}.
            \end{itemize}
        \end{itemize}
        
        \item \textbf{Gestión de roles y permisos}
    \end{enumerate}
    
\end{document}