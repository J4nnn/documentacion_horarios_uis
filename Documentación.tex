\documentclass[12pt]{article} % Tamaño de fuente 12pt

\usepackage[utf8]{inputenc} % Codificación de caracteres
\usepackage[spanish]{babel} % Idioma español
\usepackage{graphicx} % Para incluir imágenes
\usepackage{hyperref} % Para enlaces (opcional)
\usepackage{comment} % Para comentarios
\usepackage{enumitem} % Para negrita en listas

\title{Documentación del Proyecto de Horarios}
\author{Janer Alberto Vega Jácome}
\date{\today}

\begin{document}

\maketitle % Crea la portada con título, autor y fecha

\newpage
\section{Requerimientos}
\subsection{Idea general}
\noindent
El proyecto de horarios tiene como objetivo desarrollar una aplicación web que permita a los usuarios crear, gestionar y visualizar horarios de manera eficiente. Utilizando Spring Boot para el backend y Angular para el frontend, esta aplicación ofrecerá una interfaz intuitiva y funcionalidades avanzadas para la gestión de horarios.
\begin{comment}
\subsection{Equipo de Desarrollo}
\begin{table}[h!]
\centering
\begin{tabular}{|l|l|}
\hline
\textbf{Nombre} & \textbf{Rol} \\ \hline
Juan Alarcon & Desarrollador Front-End \\ \hline
Diego Medina & Desarrollador Back-End \\ \hline
Janer Vega & Documentación \\ \hline
\end{tabular}
\end{table}
\end{comment}

\subsection{Objetivos}

\subsubsection{Objetivo general}
\noindent
Desarrollar un sistema de gestión de horarios académicos para la Escuela de Ingeniería de Sistemas e Informática de la UIS que automatice y optimice la planificación, asignación y visualización de horarios y considerando las restricciones de recursos (aulas, profesores) con el fin de mejorar la eficiencia y transparencia del proceso académico.
\subsubsection{Objetivos específicos}
\begin{enumerate}[font=\bfseries] % Aplica negrita a los números
\item \textbf{Interfaz de Usuario:} Diseñar una interfaz gráfica de usuario (GUI) intuitiva y fácil de usar que permita a los usuarios (administradores, profesores y estudiantes) interactuar con el sistema de manera eficiente, proporcionando módulos específicos para la gestión de usuarios, registro académico y diseño de horarios.
\item \textbf{Gestión de Usuarios:} Implementar un módulo de gestión de usuarios que permita el registro, autenticación, autorización y administración de perfiles de usuarios (administradores, profesores, estudiantes, técnicos, auxiliares, secretarias), asignando roles y permisos adecuados para cada tipo de usuario.
\item Crear un sistema de registro que permita la inclusión, modificación, y eliminación de los elementos esenciales para la conformación de un horario académico, como son las asignaturas, profesores, aulas de clases, periodos académicos, y horas de clases.
\item Diseñar y ejecutar un algoritmo para la gestión de diseño de horarios que tome en cuenta los requerimientos de asignaturas, profesores, aulas disponibles, y secciones o grupos, proporcionando indicadores de fallas o errores como choques entre bloques de horas y solapamiento de horarios.
\item Generar reportes accesibles y detallados que permitan la visualización de horarios por profesor, aula, asignatura, y un horario general, con la capacidad de convertir y descargar estos reportes en formatos PDF y Excel.
\end{enumerate}
\subsection{Usuarios}
\noindent
El sistema de gestión de horarios está diseñado para ser utilizado por los siguientes grupos de usuarios dentro de la Escuela de Ingeniería de Sistemas de la UIS:
\subsubsection{Estudiantes}
\noindent
Los estudiantes son los principales beneficiarios del sistema. Utilizarán la aplicación para consultar sus horarios de clases, incluyendo aulas, profesores y horarios de cada materia. 
\subsubsection{Profesores}
\noindent
Los profesores también son usuarios clave del sistema. Utilizarán la aplicación para:
\begin{itemize}
    \item Consultar sus horarios de clases, incluyendo aulas y grupos asignados.
    \item Reportar cambios o ajustes en sus horarios.
    \item Visualizar la disponibilidad de aulas para programar actividades adicionales.
\end{itemize}
\subsubsection{Personal administrativo}
\noindent
El personal administrativo de la Escuela utilizará el sistema para:
\begin{itemize}
    \item Crear y gestionar los horarios académicos, asignando cursos, profesores y aulas.
    \item Realizar cambios y ajustes en los horarios según sea necesario.
    \item Generar reportes y estadísticas sobre la utilización de aulas y recursos.
    \item Gestionar los permisos de acceso de los usuarios al sistema.    
\end{itemize}


\end{document}