\documentclass[12pt]{article} % Tamaño de fuente 12pt

\usepackage[utf8]{inputenc} % Codificación de caracteres
\usepackage[spanish]{babel} % Idioma español
\usepackage{graphicx} % Para incluir imágenes
\usepackage{hyperref} % Para enlaces (opcional)
\usepackage{comment} % Para comentarios
\usepackage{enumitem} % Para negrita en listas
\usepackage[letterpaper, left=1in, right=1in, top=1in, bottom=1in]{geometry}

\title{Documentación del Proyecto de Horarios}
\author{Janer Alberto Vega Jácome}
\date{\today}

\begin{document}

\maketitle % Crea la portada con título, autor y fecha

\newpage
\section{Requerimientos}
\subsection{Idea general}
\noindent
El proyecto de horarios tiene como objetivo desarrollar una aplicación web que permita a los usuarios crear, gestionar y visualizar horarios de manera eficiente. Utilizando Spring Boot para el backend y Angular para el frontend, esta aplicación ofrecerá una interfaz intuitiva y funcionalidades avanzadas para la gestión de horarios.
\begin{comment}
\subsection{Equipo de Desarrollo}
\begin{table}[h!]
\centering
\begin{tabular}{|l|l|}
\hline
\textbf{Nombre} & \textbf{Rol} \\ \hline
Juan Alarcon & Desarrollador Front-End \\ \hline
Diego Medina & Desarrollador Back-End \\ \hline
Janer Vega & Documentación \\ \hline
\end{tabular}
\end{table}
\end{comment}

\subsection{Objetivos}
    \subsubsection{Objetivo general}
    \noindent
    Desarrollar un sistema de gestión de horarios académicos para la Escuela de Ingeniería de Sistemas e Informática de la UIS que automatice y optimice la planificación, asignación y visualización de horarios y considerando las restricciones de recursos (aulas, profesores) con el fin de mejorar la eficiencia y transparencia del proceso académico.
    \subsubsection{Objetivos específicos}
    \begin{enumerate}[font=\bfseries] % Aplica negrita a los números
        \item \textbf{Interfaz de Usuario:} Diseñar una interfaz gráfica de usuario (GUI) intuitiva y fácil de usar que permita a los usuarios (administradores, profesores y estudiantes) interactuar con el sistema de manera eficiente, proporcionando módulos específicos para la gestión de usuarios, registro académico y diseño de horarios.
        \item \textbf{Gestión de Usuarios:} Implementar un módulo de gestión de usuarios que permita el registro, autenticación, autorización y administración de perfiles de usuarios (administradores, profesores, estudiantes, técnicos, auxiliares, secretarias), asignando roles y permisos adecuados para cada tipo de usuario.
        \item \textbf{Registro Académico:} Desarrollar un sistema de registro académico que permita la gestión de información relevante para la generación de horarios, incluyendo:
        \begin{itemize}
            \item Asignaturas: Nombre, código, créditos, tipo (teórica, práctica, laboratorio).
            \item Profesores: Nombre, identificación, disponibilidad horaria, preferencias de asignaturas.
            \item Aulas: Nombre, capacidad, tipo (aula de clase, laboratorio), disponibilidad horaria.
            \item Grupos: Identificación, número de estudiantes, asignaturas asociadas.
        \end{itemize}
        \item \textbf{Algoritmo de Diseño de Horarios:} Diseñar e implementar un algoritmo eficiente y flexible para la generación automática de horarios académicos que considere:
        \begin{itemize}
            \item Restricciones de recursos: Disponibilidad de aulas y profesores en diferentes horarios.
            \item Restricciones académicas: Carga horaria de profesores, número máximo de estudiantes por grupo.
            \item Métricas de optimización: Minimizar conflictos de horarios, maximizar la utilización de recursos, equilibrar la carga horaria de profesores.
        \end{itemize}
        \item \textbf{Generación de Reportes:} Desarrollar un módulo de generación de reportes que permita visualizar y exportar horarios en diferentes formatos (PDF, Excel) para:
        \begin{itemize}
            \item Profesores: Horario individual por profesor.
            \item Aulas: Horario de ocupación de cada aula.
            \item Asignaturas: Horario de cada asignatura, incluyendo grupos y profesores.
            \item Horario General: Horario completo de la Escuela, mostrando todas las asignaturas, grupos, profesores y aulas.
        \end{itemize}
    \end{enumerate}
    
\subsection{Usuarios}
\noindent
El sistema de gestión de horarios está diseñado para ser utilizado por los siguientes grupos de usuarios dentro de la Escuela de Ingeniería de Sistemas de la UIS:
    \subsubsection{Estudiantes}
    \noindent
    Los estudiantes son los principales beneficiarios del sistema. Utilizarán la aplicación para consultar sus horarios de clases, incluyendo aulas, profesores y horarios de cada materia. 
    
    \subsubsection{Profesores}
    \noindent
    Los profesores también son usuarios clave del sistema. Utilizarán la aplicación para:
    \begin{itemize}
        \item Consultar sus horarios de clases, incluyendo aulas y grupos asignados.
        \item Reportar cambios o ajustes en sus horarios.
        \item Visualizar la disponibilidad de aulas para programar actividades adicionales.
    \end{itemize}
    
    \subsubsection{Personal administrativo}
    \noindent
    El personal administrativo de la Escuela utilizará el sistema para:
    \begin{itemize}
        \item Crear y gestionar los horarios académicos, asignando cursos, profesores y aulas.
        \item Realizar cambios y ajustes en los horarios según sea necesario.
        \item Generar reportes y estadísticas sobre la utilización de aulas y recursos.
        \item Gestionar los permisos de acceso de los usuarios al sistema.    
    \end{itemize}

\subsection{Requisitos generales}
    \subsubsection{Funcionalidad}
    \noindent
    El sistema debe permitir la creación, visualización, modificación y gestión de horarios académicos para la Escuela de Ingeniería de Sistemas e Informática de la UIS.
    \subsubsection{Usabilidad}
    \noindent
    La interfaz de usuario debe ser intuitiva, fácil de usar y accesible para todos los usuarios (administradores, profesores y estudiantes).
    \subsubsection{Rendimiento}
    \noindent
    El sistema debe ser capaz de generar horarios de manera eficiente, incluso para un gran número de asignaturas, profesores y estudiantes.
    \subsubsection{Seguridad}
    \noindent
    El sistema debe proteger la información sensible de los usuarios y garantizar que solo los usuarios autorizados puedan acceder y modificar los datos.
    \subsubsection{Escalabilidad}
    \noindent
    El sistema debe ser diseñado para adaptarse al crecimiento futuro de la Escuela, permitiendo la adición de nuevas asignaturas, profesores y estudiantes.
    \subsubsection{Compatibilidad}
    \noindent
    El sistema debe ser compatible con los navegadores web modernos (Chrome, Firefox, Safari, Edge) y funcionar en diferentes dispositivos (computadoras, tabletas, teléfonos móviles).

\subsection{Requisitos funcionales}
    \subsubsection{Gestión de usuarios}
    \begin{itemize}
        \item El sistema debe permitir el registro de nuevos usuarios (administradores, profesores, estudiantes).
        \item El sistema debe permitir la autenticación de usuarios mediante nombre de usuario y contraseña.
        \item El sistema debe asignar roles y permisos a los usuarios según su tipo (administrador, profesor, estudiante).
        \item El sistema debe permitir la modificación y eliminación de perfiles de usuario.
    \end{itemize}
    
    \subsubsection{Registro académico}
    \begin{itemize}
        \item El sistema debe permitir el registro de asignaturas, incluyendo nombre, código, créditos, prerrequisitos y tipo.
        \item El sistema debe permitir el registro de profesores, incluyendo nombre, identificación, disponibilidad horaria y preferencias de asignaturas.
        \item El sistema debe permitir el registro de aulas, incluyendo nombre, capacidad, tipo y disponibilidad horaria.
        \item El sistema debe permitir el registro de grupos, incluyendo identificación, número de estudiantes y asignaturas asociadas.
    \end{itemize}
    
    \subsubsection{Diseño de horarios}
    \begin{itemize}
        \item El sistema debe generar automáticamente horarios académicos que cumplan con las restricciones de recursos y académicas.
        \item El sistema debe permitir la modificación manual de los horarios generados.
        \item El sistema debe detectar y notificar conflictos de horarios (choques entre bloques de horas, solapamiento de horarios).
        \item El sistema debe permitir la optimización de los horarios según diferentes criterios (minimizar conflictos, maximizar la utilización de recursos, equilibrar la carga horaria de profesores).
    \end{itemize}
    
    \subsubsection{Visualización de horarios}
    \begin{itemize}
        \item El sistema debe permitir la visualización de horarios individuales por profesor.
        \item El sistema debe permitir la visualización de horarios de ocupación de aulas.
        \item El sistema debe permitir la visualización de horarios por asignatura, incluyendo grupos y profesores.
        \item El sistema debe permitir la visualización de un horario general de la Escuela.
    \end{itemize}
    
    \subsubsection{Generación de reportes}
    \begin{itemize}
        \item El sistema debe permitir la exportación de horarios en formato PDF.
        \item El sistema debe permitir la exportación de horarios en formato Excel.
    \end{itemize}

\subsection{Información de autoría}
\begin{itemize}
    \item Este proyecto es desarrollado por estudiantes de la Escuela de Ingeniería de Sistemas e Informática de la Universidad Industrial de Santander (UIS), como parte de una auxiliatura académica.
    \item Los derechos de autor del código fuente y la documentación del proyecto pertenecen a los estudiantes desarrolladores \textit{(Por ahora jajajaja)}.
    \item El proyecto se distribuye bajo una licencia de código abierto \textit{(debemos escoger una)}.
\end{itemize}

\subsection{Alcances y limitaciones}
    \subsubsection{Alcances}
    \begin{itemize}
        \item El sistema se enfocará en la gestión de horarios académicos para la Escuela de Ingeniería de Sistemas e Informática de la UIS.
        \item El sistema permitirá la generación automática de horarios, considerando restricciones y preferencias.
        \item El sistema ofrecerá diferentes vistas de horarios y la posibilidad de exportarlos en varios formatos.
    \end{itemize}
    
    \subsubsection{Limitaciones}
    \begin{itemize}
        \item El sistema no gestionará otros aspectos de la vida académica, como calificaciones, asistencia o inscripciones.
        \item El sistema no se integrará inicialmente con otros sistemas de la universidad.
        \item El algoritmo de diseño de horarios puede no encontrar soluciones óptimas en todos los casos, especialmente si las restricciones son muy complejas o contradictorias.
    \end{itemize}

\subsection{Herramientas de desarrollo}
\begin{enumerate}[font=\bfseries]
    \item \textbf{IntelliJ IDEA Ultimate}
    \begin{itemize}
        \item \textbf{Tipo:} Entorno de desarrollo integrado (IDE)
        \item \textbf{Justificación:} Para desarrollar el backend de la aplicación en Spring Boot, aprovechando sus herramientas de desarrollo Java y su integración con frameworks de Spring.
    \end{itemize}
    
    \item \textbf{Visual Studio Code (VS Code)}
    \begin{itemize}
        \item \textbf{Tipo:} Editor de código fuente
        \item \textbf{Justificación:} Para desarrollar el frontend de la aplicación en Angular, utilizando sus extensiones para Angular y TypeScript, así como sus herramientas de depuración y pruebas.
    \end{itemize}
    
    \item \textbf{MySQL Workbench}
    \begin{itemize}
        \item \textbf{Tipo:} Herramienta de administración de bases de datos
        \item \textbf{Justificación:} Para diseñar y administrar la base de datos MySQL donde se almacenarán los datos de horarios, asignaturas, profesores, aulas, etc.
    \end{itemize}
    
    \item \textbf{GitHub}
    \begin{itemize}
        \item \textbf{Tipo:} Plataforma de alojamiento de código y colaboración
        \item \textbf{Justificación:} Para alojar el código fuente del proyecto, facilitar la colaboración entre los miembros del equipo y llevar un seguimiento de los cambios en el código a lo largo del tiempo.
    \end{itemize}
    
    \item \textbf{Tailscale}
    \begin{itemize}
        \item \textbf{Tipo:} Red privada virtual (VPN)
        \item \textbf{Justificación:} Facilitar la colaboración entre los miembros del equipo de desarrollo, permitiéndoles compartir recursos y trabajar en el proyecto de forma remota como si estuvieran en la misma red local.
    \end{itemize}
    
    \item \textbf{Swagger UI}
    \begin{itemize}
        \item \textbf{Tipo:} Herramienta de visualización y documentación de APIs REST
        \item \textbf{Justificación:} Permitir a los desarrolladores probar los endpoints de la API directamente desde la interfaz de usuario, agilizando el proceso de desarrollo y depuración. \\
        Proporcionar una documentación clara y completa de la API REST del sistema, facilitando su uso por parte del desarrollador frontend.
    \end{itemize}
    
    \item \textbf{Ubuntu Server}
    \begin{itemize}
        \item \textbf{Tipo:} Sistema operativo de servidor basado en Linux
        \item \textbf{Justificación:} Alojar el backend en un servidor dedicado, así el sistema está disponible las 24 horas del día, los 7 días de la semana, lo que garantiza que el desarrollador frontend podrá consumir las APIs en cualquier momento, lo que facilita su trabajo.
    \end{itemize}
\end{enumerate}

\end{document}